\section{Anhang}

\subsection{Beschreibung der Use-Cases}

% Registrieren
\subsubsection{Registrieren}\label{subsubsec:uc_reg}
\noindent \textbf{use case} Registrieren \newline
\indent \textbf{actors} \newline
\indent \indent Benutzer, Backend \newline
\indent \textbf{precondition} \newline
\indent \indent - \newline
\indent \textbf{main flow}
\begin{enumerate}[labelwidth=0pt,leftmargin=39pt,noitemsep,topsep=0pt,parsep=0pt,partopsep=0pt]
\item Der Benutzer klickt den Link 'Registrieren' und öffnet damit den Registrierungs-Dialog.
\item Der Benutzer füllt alle Formularfelder aus und klickt den Registrieren-Button.
\item Das System sendet eine Email mit einem Bestätigungslink an die angegebene Email-Adresse.
\item Der Benutzer ruft die Url des Bestätigungs-Links auf
\end{enumerate}
\indent \indent \textbf{postcondition} \newline
\indent \indent Der Benutzer ist registriert \newline
\indent \textbf{exceptional flow} Email bereits eingetragen \newline
\indent \indent Ein Benutzer ist mit der angegebenen Email-Adresse bereits registriert \newline
\indent \textbf{postcondition}
\begin{itemize}[label={},labelwidth=0pt,leftmargin=24pt,noitemsep,topsep=0pt,parsep=0pt,partopsep=0pt]
\item Der Registrierungs-Dialog bleibt geöffnet und das Problem wird angezeigt
\end{itemize}
\indent \indent \textbf{exceptional flow} Passwort zu kurz
\begin{itemize}[label={},labelwidth=0pt,leftmargin=24pt,noitemsep,topsep=0pt,parsep=0pt,partopsep=0pt]
\item Das Passwort muss mindestens 8 Zeichen lang sein
\end{itemize}
\indent \indent \textbf{postcondition}
\begin{itemize}[label={},labelwidth=0pt,leftmargin=24pt,noitemsep,topsep=0pt,parsep=0pt,partopsep=0pt]
\item Der Registrierungs-Dialog bleibt geöffnet und das Problem wird angezeigt
\end{itemize}
\noindent \textbf{end} Registrieren \newline

% Anmelden
\subsubsection{Anmelden}\label{subsubsec:uc_anm}
\noindent \textbf{use case} Anmelden \newline
\indent \textbf{actors} \newline
\indent \indent Benutzer, Backend \newline
\indent \textbf{precondition} \newline
\indent \indent Der Benutzer ist nicht angemeldet \newline
\indent \textbf{main flow}
\begin{enumerate}[labelwidth=0pt,leftmargin=39pt,noitemsep,topsep=0pt,parsep=0pt,partopsep=0pt]
\item Der Benutzer klickt den Link 'Anmelden' und öffnet damit den Anmelde-Dialog.
\item Der Benutzer füllt alle Formularfelder aus und klickt den Anmelden-Button.
\end{enumerate}
\indent \indent \textbf{postcondition} \newline
\indent \indent Der Benutzer ist angemeldet \newline
\indent \textbf{exceptional flow} ungültige Anmeldedaten \newline
\indent \indent es existiert kein Benutzer für die angegebenen Anmeldedaten \newline
\indent \textbf{postcondition}
\begin{itemize}[label={},labelwidth=0pt,leftmargin=24pt,noitemsep,topsep=0pt,parsep=0pt,partopsep=0pt]
\item Der Registrierungs-Dialog bleibt geöffnet und das Problem wird angezeigt
\end{itemize}
\noindent \textbf{end} Anmelden \newline

% Lokale Karte navigieren
\subsubsection{Lokale Karte navigieren (Positionierung des 'User'-Markers)}\label{subsubsec:uc_locmapnav}
\noindent \textbf{use case} Lokale Karte navigieren \newline
\indent \textbf{actors} \newline
\indent \indent Benutzer \newline
\indent \textbf{precondition} \newline
\indent \indent - \newline
\indent \textbf{main flow}
\begin{enumerate}[labelwidth=0pt,leftmargin=39pt,noitemsep,topsep=0pt,parsep=0pt,partopsep=0pt]
\item Der Benutzer verschiebt den sichtbaren Kartenausschnitt indem er den Mauszeiger bei gedrückter linker Maustaste zieht.
\item Der Benutzer markiert eine gewünschte Position indem er mit einem einfachen linken Mausklick auf die Karte klickt
\end{enumerate}
\indent \indent \textbf{alternative flow}
\begin{itemize}[label={},labelwidth=0pt,leftmargin=24pt,noitemsep,topsep=0pt,parsep=0pt,partopsep=0pt]
\item Der Benutzer zieht den blauen 'User'-Marker mit gedrückter linker Maustaste zu einer gewünschten Position.
\end{itemize}
\indent \indent \textbf{alternative flow} \newline
\indent \indent Der blaue 'User'-Marker wird ber GPS positioniert \newline
\indent \textbf{postcondition}
\begin{itemize}[label={},labelwidth=0pt,leftmargin=24pt,noitemsep,topsep=0pt,parsep=0pt,partopsep=0pt]
\item Die Karte zentriert um die markierte Stelle und der blaue 'User'-Marker wird an dieser Position angezeigt.
\end{itemize}
\indent \indent \textbf{exceptional flow} fehlende Kacheln
\begin{itemize}[label={},labelwidth=0pt,leftmargin=24pt,noitemsep,topsep=0pt,parsep=0pt,partopsep=0pt]
\item das lokale System ist offline und für die gewünschte Position existieren keine vorgeladenen Kacheln
\end{itemize}
\indent \indent \textbf{postcondition}
\begin{itemize}[label={},labelwidth=0pt,leftmargin=24pt,noitemsep,topsep=0pt,parsep=0pt,partopsep=0pt]
\item Der Benutzer kann den 'User'-Marker nur an Stellen positionieren, für welche Kacheln vorgeladen wurden.
\end{itemize}
\noindent \textbf{end} Lokale Karte navigieren \newline

% Entfernte Karte navigieren
\subsubsection{Zeichnen auf entfernten Karten}\label{subsubsec:uc_remmapnav}
\noindent \textbf{use case} Zeichnen auf entfernten Karten \newline
\indent \textbf{actors} \newline
\indent \indent Benutzer, Peer, Comm \newline
\indent \textbf{precondition} \newline
\indent \indent Erlaubnis des entfernten Benutzers \newline
\indent \textbf{main flow}
\begin{enumerate}[labelwidth=0pt,leftmargin=39pt,noitemsep,topsep=0pt,parsep=0pt,partopsep=0pt]
\item Der Benutzer klickt auf den 'Spur'-Link im Toolbar-Popup eines 'Peer'-Markers von einem anderen Benutzer
\end{enumerate}
\indent \indent \textbf{postcondition}
\begin{itemize}[label={},labelwidth=0pt,leftmargin=24pt,noitemsep,topsep=0pt,parsep=0pt,partopsep=0pt]
\item Die Position des blauen 'User'-Markers auf der Karte des gewählten Peers und dessen folgende Positions-Änderungen werden als Linie auf der eigenen Karte angezeigt.
\end{itemize}
\noindent \textbf{end} Zeichnen auf entfernten Karten \newline

% PoI markieren
\subsubsection{PoI markieren}\label{subsubsec:uc_poinew}
\noindent \textbf{use case} PoI markieren \newline
\indent \textbf{actors} \newline
\indent \indent Benutzer \newline
\indent \textbf{precondition} \newline
\indent \indent - \newline
\indent \textbf{main flow}
\begin{enumerate}[labelwidth=0pt,leftmargin=39pt,noitemsep,topsep=0pt,parsep=0pt,partopsep=0pt]
\item Der Benutzer positioniert den 'User'-Marker an der gewünschten Stelle.
\item Im Marker-Toolbar-Popup klickt er auf 'PoI Eintragen'.
\item Es öffnet sich ein Dialog mit Eingabe-Feldern für Text und Anhang zum Eintrag.
\item Der Benutzer füllt die entsprechenden Felder aus und klickt auf den Button 'Sichern'.
\end{enumerate}
\indent \indent \textbf{alternative flow} \newline
\indent \indent Der Benutzer öffnet den Eintrags-Dialog mit einem Doppelklick \newline
\indent \textbf{postcondition}
\begin{itemize}[label={},labelwidth=0pt,leftmargin=24pt,noitemsep,topsep=0pt,parsep=0pt,partopsep=0pt]
\item Ein roter PoI-Marker wird auf der Karte angezeigt. In einem darüberliegenden Popup werden die im Dialog-Formular eingetragenen Daten angezeigt.  
\end{itemize}
\indent \indent \textbf{exceptional flow} Format des Media-Anhangs nicht unterstützt
\begin{itemize}[label={},labelwidth=0pt,leftmargin=24pt,noitemsep,topsep=0pt,parsep=0pt,partopsep=0pt]
\item Der Benutzer möchte ein nicht unterstütztes Media-Format einer externen Quelle an den Eintrag anhängen.
\end{itemize}
\indent \indent \textbf{postcondition}
\begin{itemize}[label={},labelwidth=0pt,leftmargin=24pt,noitemsep,topsep=0pt,parsep=0pt,partopsep=0pt]
\item Der Eingabe-Dialog bleibt geöffnet und der Fehler wird angezeigt.
\end{itemize}
\indent \indent \textbf{exceptional flow} PoI bereits eingetragen
\begin{itemize}[label={},labelwidth=0pt,leftmargin=24pt,noitemsep,topsep=0pt,parsep=0pt,partopsep=0pt]
\item Der Benutzer trägt innerhalb eines 3m-Umkreises eines exisitierenden PoI's einen neuen PoI ein
\end{itemize}
\indent \indent \textbf{postcondition}
\begin{itemize}[label={},labelwidth=0pt,leftmargin=24pt,noitemsep,topsep=0pt,parsep=0pt,partopsep=0pt]
\item Der Eintrag des Benutzers wird bei beim nächsten Datenabgleich im Online-Modus als Kommentar je nach Sortierungskriterium an erster oder letzter Stelle des  existierenden PoI-Eintrags angefügt
\end{itemize}
\noindent \textbf{end} PoI markieren \newline

% PoI kommentieren
\subsubsection{PoI kommentieren}\label{subsubsec:uc_poinote}
\noindent \textbf{use case} PoI kommentieren \newline
\indent \textbf{actors} \newline
\indent \indent Benutzer \newline
\indent \textbf{precondition} \newline
\indent \indent PoI-Popup geöffnet \newline
\indent \textbf{main flow}
\begin{enumerate}[labelwidth=0pt,leftmargin=39pt,noitemsep,topsep=0pt,parsep=0pt,partopsep=0pt]
\item Der Benutzer klickt auf den 'Kommentieren'-Link im PoI-Marker-Popup und öffnet damit einen Eingabe-Dialog.
\item Der Benutzer füllt die entsprechenden Eingabe-Felder aus und klickt den 'Eintragen'-Button.
\end{enumerate}
\indent \indent \textbf{postcondition}
\begin{itemize}[label={},labelwidth=0pt,leftmargin=24pt,noitemsep,topsep=0pt,parsep=0pt,partopsep=0pt]
\item Der Kommentar des Benutzers wird an letzter Stelle der PoI-Kommentarliste angefügt
\end{itemize}
\noindent \textbf{end} PoI kommentieren \newline

% Poi merken
\subsubsection{Poi merken}\label{subsubsec:uc_poibookmark}
\noindent \textbf{use case} Poi merken \newline
\indent \textbf{actors} \newline
\indent \indent Benutzer \newline
\indent \textbf{precondition} \newline
\indent \indent - \newline
\indent \textbf{main flow}
\begin{enumerate}[labelwidth=0pt,leftmargin=39pt,noitemsep,topsep=0pt,parsep=0pt,partopsep=0pt]
\item Der Benutzer klickt auf den 'Bookmark'-Link im Toolbar-Popup des 'Poi'-Markers.
\end{enumerate}
\indent \indent \textbf{postcondition}
\begin{itemize}[label={},labelwidth=0pt,leftmargin=24pt,noitemsep,topsep=0pt,parsep=0pt,partopsep=0pt]
\item Der markierte Ort ist in der Lesezeichen-Liste des Benutzers eingetragen
\item Der Benutzer sieht bei künftigen Klicks auf das Toolbar-Icon die gespeicherte Notiz
\end{itemize}
\noindent \textbf{end} Ort merken \newline

% Ort merken
\subsubsection{Ort merken}\label{subsubsec:uc_locnote}
\noindent \textbf{use case} Ort merken \newline
\indent \textbf{actors} \newline
\indent \indent Benutzer \newline
\indent \textbf{precondition} \newline
\indent \indent - \newline
\indent \textbf{main flow}
\begin{enumerate}[labelwidth=0pt,leftmargin=39pt,noitemsep,topsep=0pt,parsep=0pt,partopsep=0pt]
\item Der Benutzer klickt auf den 'Ort merken'-Link im Toolbar-Popup des 'User'-Markers.
\end{enumerate}
\indent \indent \textbf{postcondition}
\begin{itemize}[label={},labelwidth=0pt,leftmargin=24pt,noitemsep,topsep=0pt,parsep=0pt,partopsep=0pt]
\item Der markierte Ort ist in der Lesezeichen-Liste des Benutzers eingetragen
\item Der Benutzer sieht bei künftigen Klicks auf das Toolbar-Icon die gespeicherte Notiz
\end{itemize}
\noindent \textbf{end} Ort merken \newline

% Wahrnehmung eines anderen Benutzers beantragen
\subsubsection{Wahrnehmung eines anderen Benutzers beantragen}\label{subsubsec:uc_reqmefollowpeer}
\noindent \textbf{use case} Wahrnehmung eines anderen Benutzers beantragen \newline
\indent \textbf{actors} \newline
\indent \indent Benutzer, Backend, Comm, Peer \newline
\indent \textbf{precondition}
\begin{enumerate}[labelwidth=0pt,leftmargin=39pt,noitemsep,topsep=0pt,parsep=0pt,partopsep=0pt]
\item Der Benutzer befindet sich in der Ansicht 'Home'
\item Der Benutzer folgt dem ausgewählten anderen Benutzer noch nicht
\end{enumerate}
\indent \indent \textbf{main flow}
\begin{enumerate}[labelwidth=0pt,leftmargin=39pt,noitemsep,topsep=0pt,parsep=0pt,partopsep=0pt]
\item Der Benutzer markiert einen anderen Benutzer aus der Liste 'Ich folge nicht'
\item Der Benutzer sendet den Antrag auf Wahrnehmung mit einem Klick auf den 'Sichern'-Button an das Backend
\end{enumerate}
\indent \indent \textbf{postcondition}
\begin{itemize}[labelwidth=0pt,leftmargin=39pt,noitemsep,topsep=0pt,parsep=0pt,partopsep=0pt]
\item Das Backend sendet die Anfrage über Comm an den anderen Benutzer
\item Der ausgewählte andere Benutzer ist jetzt unter 'Ich möchte folgen' gelistet
\end{itemize}
\noindent \textbf{end} Wahrnehmung eines anderen Benutzers beantragen \newline

% Wahrnehmung durch einen anderen Benutzer gestatten
\subsubsection{Wahrnehmung durch einen anderen Benutzer gestatten}\label{subsubsec:uc_grantpeerfollowme}
\noindent \textbf{use case} Wahrnehmung durch einen anderen Benutzer gestatten \newline
\indent \textbf{actors} \newline
\indent \indent Benutzer, Backend, Comm, Peer \newline
\indent \textbf{precondition}
\begin{enumerate}[labelwidth=0pt,leftmargin=39pt,noitemsep,topsep=0pt,parsep=0pt,partopsep=0pt]
\item Der Benutzer befindet sich in der Ansicht 'Home'
\item Ein anderer Benutzer hat eine entsprechende Anfrage gestellt und ist unter 'Möchten mir folgen' gelistet
\end{enumerate}
\indent \indent \textbf{main flow}
\begin{enumerate}[labelwidth=0pt,leftmargin=39pt,noitemsep,topsep=0pt,parsep=0pt,partopsep=0pt]
\item Der Benutzer markiert die 'erlauben'-Checkbox in der Zeile mit dem anderen Benutzernamen
\item Der Benutzer sendet die Erlaubnis zur Wahrnehmung mit einem Klick auf den 'Sichern'-Button an das Backend
\end{enumerate}
\indent \indent \textbf{postcondition}
\begin{itemize}[labelwidth=0pt,leftmargin=39pt,noitemsep,topsep=0pt,parsep=0pt,partopsep=0pt]
\item Das Backend sendet die Erlaubnis über Comm an den ausgewählten anderen Benutzer
\item Der ausgewählte andere Benutzer ist jetzt unter 'Folgen mir' gelistet
\item Es ist ab sofort möglich, dem anderen Benutzer Nachrichten zu schicken.
\end{itemize}
\noindent \textbf{end} Wahrnehmung durch einen anderen Benutzer gestatten \newline

% Wahrnehmung durch einen anderen Benutzer verweigern
\subsubsection{Wahrnehmung durch einen anderen Benutzer gestatten}\label{subsubsec:uc_denypeerfollowme}
\noindent \textbf{use case} Wahrnehmung durch einen anderen Benutzer verweigern \newline
\indent \textbf{actors} \newline
\indent \indent Benutzer, Backend, Comm, Peer \newline
\indent \textbf{precondition}
\begin{enumerate}[labelwidth=0pt,leftmargin=39pt,noitemsep,topsep=0pt,parsep=0pt,partopsep=0pt]
\item Der Benutzer befindet sich in der Ansicht 'Home'
\item Ein anderer Benutzer hat eine entsprechende Anfrage gestellt und ist unter 'Möchten mir folgen' gelistet
\end{enumerate}
\indent \indent \textbf{main flow}
\begin{enumerate}[labelwidth=0pt,leftmargin=39pt,noitemsep,topsep=0pt,parsep=0pt,partopsep=0pt]
\item Der Benutzer markiert die 'verweigern'-Checkbox in der Zeile mit dem anderen Benutzernamen
\item Der Benutzer sendet die Verweigerung zur Wahrnehmung mit einem Klick auf den 'Sichern'-Button an das Backend
\end{enumerate}
\indent \indent \textbf{postcondition}
\begin{itemize}[labelwidth=0pt,leftmargin=39pt,noitemsep,topsep=0pt,parsep=0pt,partopsep=0pt]
\item Das Backend sendet die Verweigerung über Comm an den ausgewählten anderen Benutzer
\item Der ausgewählte andere Benutzer ist jetzt nicht mehr unter 'Möchten mir folgen' gelistet
\end{itemize}
\noindent \textbf{end} Wahrnehmung durch einen anderen Benutzer verweigern \newline

% Notiz zu anderem Benutzer erstellen
\subsubsection{Notiz zu anderem Benutzer erstellen}\label{subsubsec:uc_usernote}
\noindent \textbf{use case} Notiz zu anderem Benutzer erstellen \newline
\indent \textbf{actors} \newline
\indent \indent Benutzer \newline
\indent \textbf{precondition} \newline
\indent \indent - \newline
\indent \textbf{main flow}
\begin{enumerate}[labelwidth=0pt,leftmargin=39pt,noitemsep,topsep=0pt,parsep=0pt,partopsep=0pt]
\item Der Benutzer klickt auf den 'Notiz-hinzufügen'-Link im Toolbar-Popup des Markers eines anderen Benutzers.
\item Der Benutzer schreibt eine Notiz ins Textfeld
\item Der Benutzer schließt das Fenster oder öffnet ein Popup eines beliebigen anderen Markers um die Notiz zu sichern
\end{enumerate}
\indent \indent \textbf{postcondition}
\begin{itemize}[label={},labelwidth=0pt,leftmargin=24pt,noitemsep,topsep=0pt,parsep=0pt,partopsep=0pt]
\item Der Notiz-Text ist gesichert und kann durch erneuten Klick  auf den 'Notiz-hinzufügen'-Link angezeigt werden.
\end{itemize}
\noindent \textbf{end} Notiz zu anderem Benutzer erstellen \newline

% P2P-Chat
\subsubsection{P2P-Chat}\label{subsubsec:uc_chatp2p}
\noindent \textbf{use case} P2P-Chat \newline
\indent \textbf{actors} \newline
\indent \indent Benutzer \newline
\indent \textbf{precondition} \newline
\indent \indent P2P-Chat-Popup eines 'Peer'-Markers geöffnet \newline
\indent \textbf{main flow}
\begin{enumerate}[labelwidth=0pt,leftmargin=39pt,noitemsep,topsep=0pt,parsep=0pt,partopsep=0pt]
\item Der Benutzer schreibt den Nachrichten-Text in das Eingabe-Textfeld und schickt die Nachricht mit drücken der EingabeTaste ab.
\end{enumerate}
\indent \indent \textbf{postcondition}
\begin{itemize}[label={},labelwidth=0pt,leftmargin=24pt,noitemsep,topsep=0pt,parsep=0pt,partopsep=0pt]
\item Die gesendete Nachricht erscheint in einer Sprechblase mit Pfeil nach links im P2P-Chat-Popup.
\item Der angeschriebene andere Benutzer wird auf die eingegangene Nachricht aufmerksam gemacht
\item Die gesendete Nachricht erscheint in einer Sprechblase mit Pfeil nach rechts im P2P-Chat-Popup des angeschriebenen Benutzers.
\end{itemize}
\noindent \textbf{end} P2P-Chat \newline

% Konferenz
\subsubsection{Konferenz}\label{subsubsec:uc_chatconf}
\noindent \textbf{use case} Konferenz \newline
\indent \textbf{actors} \newline
\indent \indent Benutzer \newline
\indent \textbf{precondition} \newline
\indent \indent Der Benutzer befinet sich in der Ansicht 'Konferenz' \newline
\indent \textbf{main flow}
\begin{enumerate}[labelwidth=0pt,leftmargin=39pt,noitemsep,topsep=0pt,parsep=0pt,partopsep=0pt]
\item Der Benutzer schreibt den Nachrichten-Text in das Eingabe-Textfeld und schickt die Nachricht mit drücken der EingabeTaste ab.
\end{enumerate}
\indent \indent \textbf{postcondition}
\begin{itemize}[label={},labelwidth=0pt,leftmargin=24pt,noitemsep,topsep=0pt,parsep=0pt,partopsep=0pt]
\item Die gesendete Nachricht erscheint in einer Sprechblase mit Pfeil nach links in der Konferenz-Chat-Ansicht.
\item Die gesendete Nachricht erscheint in einer Sprechblase mit Pfeil nach rechts in der Konferenz-Chat-Ansicht aller wahrnehmungsberechtigten Benutzer
\end{itemize}
\noindent \textbf{end} Konferenz

% Internetverbindungs-Status
\subsubsection{Internetverbindungs-Status}\label{subsubsec:uc_watchconnstate}
\noindent \textbf{use case} Internetverbindungs-Status \newline
\indent \textbf{actors} \newline
\indent \indent Benutzer \newline
\indent \textbf{precondition} \newline
\indent \indent - \newline
\indent \textbf{main flow}
\begin{enumerate}[labelwidth=0pt,leftmargin=39pt,noitemsep,topsep=0pt,parsep=0pt,partopsep=0pt]
\item Der Status der Internetverbindung ändert sich
\end{enumerate}
\indent \indent \textbf{postcondition}
\begin{itemize}[label={},labelwidth=0pt,leftmargin=24pt,noitemsep,topsep=0pt,parsep=0pt,partopsep=0pt]
\item im Offline-Status werden die Daten und Aktionen lokal gespeichert und beim nächsten Verbindungsaufbau mit dem Backend synchronisiert
\end{itemize}
\noindent \textbf{end} Internetverbindungs-Status

% Berechtigung zur Wahrnehmung eines anderen Benutzers erhalten
\subsubsection{Berechtigung zur Wahrnehmung eines anderen Benutzers erhalten}\label{subsubsec:uc_watchgrantmefollowpeer}
\noindent \textbf{use case} Berechtigung zur Wahrnehmung eines anderen Benutzers erhalten \newline
\indent \textbf{actors} \newline
\indent \indent Wahrnehmer, Backend, Comm, Peer \newline
\indent \textbf{precondition}
\begin{enumerate}[labelwidth=0pt,leftmargin=39pt,noitemsep,topsep=0pt,parsep=0pt,partopsep=0pt]
\item das lokale System ist online
\item Der Benutzer hat eine Anfrage zur Wahrnehmung an einen anderen Benutzer gestellt
\end{enumerate}
\indent \indent \textbf{main flow}
\begin{enumerate}[labelwidth=0pt,leftmargin=39pt,noitemsep,topsep=0pt,parsep=0pt,partopsep=0pt]
\item Das Backend sendet über Comm die Erteilung der Erlaubnis zur Wahrnehmung eines anderen Benutzers
\end{enumerate}
\indent \indent \textbf{postcondition}
\begin{itemize}[label={},labelwidth=0pt,leftmargin=24pt,noitemsep,topsep=0pt,parsep=0pt,partopsep=0pt]
\item Der ausgewählte andere Benutzer ist jetzt unter 'Ich folge' gelistet
\item In der Ansicht 'People Of Interest' wird der andere Benutzer gelistet und es wird ein Link zu seiner aktuellen Position angezeigt.
\item Ist diese Position im Nahbereich, wird ein gelber 'Peer'-Marker an dieser Position angezeigt
\item Es ist ab sofort möglich, in Echtzeit mit dem anderen Benutzer zu Kommunizieren
\end{itemize}
\noindent \textbf{end} Berechtigung zur Wahrnehmung eines anderen Benutzers erhalten

% Position/Bewegung anderer Benutzer
\subsubsection{Position/Bewegung anderer Benutzer}\label{subsubsec:uc_watchposofpeer}
\noindent \textbf{use case} Position/Bewegung anderer Benutzer \newline
\indent \textbf{actors} \newline
\indent \indent Wahrnehmer, Comm, Peer \newline
\indent \textbf{precondition} \newline
\indent \indent das lokale System ist online \newline
\indent \textbf{main flow}
\begin{enumerate}[labelwidth=0pt,leftmargin=39pt,noitemsep,topsep=0pt,parsep=0pt,partopsep=0pt]
\item ein anderer Benutzer befindet sich im Nahbereich.
\end{enumerate}
\indent \indent \textbf{postcondition}
\begin{itemize}[label={},labelwidth=0pt,leftmargin=24pt,noitemsep,topsep=0pt,parsep=0pt,partopsep=0pt]
\item Ein 'Peer'-Marker ist auf der lokalen Karte an der Position zu sehen, an welcher der 'User'-Marker auf der entfernten Karte des anderen Benutzers positioniert ist.
\item Änderungen werden in Echtzeit bernommen.
\end{itemize}
\noindent \textbf{end} Position/Bewegung anderer Benutzer

% Poi-Markierung/Kommentar anderer Benutzer
\subsubsection{Poi-Markierung/Kommentar anderer Benutzer}\label{subsubsec:uc_watchpoinewornotebypeer}
\noindent \textbf{use case} Poi-Markierung/Kommentar anderer Benutzer \newline
\indent \textbf{actors} \newline
\indent \indent Wahrnehmer, Comm, Peer \newline
\indent \textbf{precondition} \newline
\indent \indent das lokale System ist online \newline
\indent \textbf{main flow}
\begin{enumerate}[labelwidth=0pt,leftmargin=39pt,noitemsep,topsep=0pt,parsep=0pt,partopsep=0pt]
\item Ein wahrnehmbarer anderer Benutzer bearbeitet einen PoI
\end{enumerate}
\indent \indent \textbf{alternative flow}
\begin{enumerate}[labelwidth=0pt,leftmargin=39pt,noitemsep,topsep=0pt,parsep=0pt,partopsep=0pt]
\item Der PoI wurde editiert als der jetzt angemeldete Benutzer offline war
\end{enumerate}
\indent \indent \textbf{postcondition}
\begin{itemize}[label={},labelwidth=0pt,leftmargin=24pt,noitemsep,topsep=0pt,parsep=0pt,partopsep=0pt]
\item Ein visuelles und/oder akkustisches Signal weist den Benutzer auf die Änderungen am PoI hin (grüner Pfeil blinkt)
\item In der 'Points of Interest'-Ansicht blinkt der entsprechende Eintrag und der Benutzer kann über einen Link direkt zum PoI navigieren und die Änderung sehen.
\end{itemize}
\noindent \textbf{end} Poi-Markierung/Kommentar anderer Benutzer

% P2P-Chat/Konferenz-Nachricht anderer Benutzer 
\subsubsection{P2P-Chat/Konferenz-Nachricht anderer Benutzer}\label{subsubsec:uc_watchmsgbypeer}
\noindent \textbf{use case} P2P-Chat/Konferenz-Nachricht anderer Benutzer \newline
\indent \textbf{actors} \newline
\indent \indent Wahrnehmer, Comm, Peer \newline
\indent \textbf{precondition} \newline
\indent \indent das lokale System ist online \newline
\indent \textbf{main flow}
\begin{enumerate}[labelwidth=0pt,leftmargin=39pt,noitemsep,topsep=0pt,parsep=0pt,partopsep=0pt]
\item Ein wahrgenommener anderer Benutzer sendet eine P2P-Chat- oder Konferenz-Nachricht
\end{enumerate}
\indent \indent \textbf{alternative flow}
\begin{enumerate}[labelwidth=0pt,leftmargin=39pt,noitemsep,topsep=0pt,parsep=0pt,partopsep=0pt]
\item Die P2P-Chat- oder Konferenz-Nachricht wurde gesendet als der jetzt angemeldete Benutzer offline war
\end{enumerate}
\indent \indent \textbf{postcondition}
\begin{itemize}[label={},labelwidth=0pt,leftmargin=24pt,noitemsep,topsep=0pt,parsep=0pt,partopsep=0pt]
\item Ein visuelles und/oder akkustisches Signal weist den Benutzer auf den Empfang einer Nachricht des anderen Benutzers hin (Konferenz-Menu oder 'Peer'-Marker des anderen Benutzers blinken)
\item Die Nachricht wird als Sprechblase mit einem rechtsgerichteten Pfeil angezeigt.
\end{itemize}
\noindent \textbf{end} P2P-Chat/Konferenz-Nachricht anderer Benutzer

\newpage
