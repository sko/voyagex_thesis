\begin{center}
\textbf{Zusammenfassung}
\end{center}

\noindent In dieser Arbeit wird die Entwicklung einer Web-Anwendung zur gruppenbasierten Bearbeitung von interaktiven Karten beschrieben.\\
Dabei soll neue Funktionalität für die bereits produktiv eingesetzte Anwendung PUMAS-Voyage in einem vergleichbaren Anwendungskontext entwickelt werden. Insbesondere sollen die 3 neuen Anforderungen
\begin{itemize}
\item Synchrone Interaktion
\item Bereitstellung einer geographischen Basis-Karte und Bearbeitung darüberliegender Darstellungs-Schichten im Offline-Modus
\item Implementierung einer Versionsverwaltung zur Umsetzung verteilter Kooperation
\end{itemize}
implementiert werden.

  \vspace{2cm}

\begin{center}
\textbf{Abstract}
\end{center}

\noindent This bachelor thesis's objectives are to describe the Development of a Web-Application for group-based
Collaboration on interactive Maps. 